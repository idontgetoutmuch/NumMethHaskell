\documentclass[presentation]{beamer}
\usepackage[utf8]{inputenc}
\usepackage[T1]{fontenc}
\usepackage{fixltx2e}
\usepackage{graphicx}
\usepackage{grffile}
\usepackage{longtable}
\usepackage{wrapfig}
\usepackage{rotating}
\usepackage[normalem]{ulem}
\usepackage{amsmath}
\usepackage{textcomp}
\usepackage{amssymb}
\usepackage{capt-of}
\usepackage{hyperref}
\usepackage{listings}
\usepackage{color}
\usepackage{verse}
\RequirePackage{fancyvrb}
\DefineVerbatimEnvironment{verbatim}{Verbatim}{fontsize=\scriptsize}
\usepackage[style=alphabetic]{biblatex}
\usetheme{Frankfurt}
\author{Dominic Steinitz}
\date{Wednesday 27 September 17}
\title{Hacking on GHC: A Worm's Eye View}
\hypersetup{
 pdfauthor={Dominic Steinitz},
 pdfkeywords={},
 pdflang={English}}

\usepackage{dramatist}

\begin{document}

\maketitle
\begin{frame}{Outline}
\tableofcontents
\end{frame}

\section{Introduction}

\begin{frame}{Apollo 8 launched on December 21, 1968}

\StageDir{03:17:45:17 (Dec. 25, 1968, 6:36 a.m. UTC)}

\begin{drama}
  \Character{Jim Lovell (Commander Module Pilot)}{jim}
  \Character{Ken Mattingly (CAPCOM)}{ken}

  \jimspeaks: Roger. Do you wish me to reinitialize the W-matrix at this time?
\end{drama}

\StageDir{03:17:45:26}

\begin{drama}
  \Character{Jim Lovell (Commander Module Pilot)}{jim}
  \Character{Ken Mattingly (CAPCOM)}{ken}

  \kenspeaks: Affirmative, Apollo 8
\end{drama}

\section{Introducing the Reverend Bayes}
\end{frame}

\begin{frame}{Game}

  \begin{block}{Game}
    \begin{itemize}
    \item I select a number at random from a normal distribution.
    \item At time 1 I give you some information: the number with added noise.
    \item At time 2 I give you more information: the same number but with different added noise.
    \item And so on $\ldots$
    \end{itemize}
  \end{block}

\end{frame}

\begin{frame}{Bayes' Theorem}

  $$
  \mathbb{P}(A \,|\, B) \triangleq \frac{\mathbb{P}(A \cap B)}{\mathbb{P}(B)}
  $$

  Also

  $$
  \mathbb{P}(B \,|\, A) \triangleq \frac{\mathbb{P}(A \cap B)}{\mathbb{P}(A)}
  $$

  Thus

  $$
  \mathbb{P}(A \,|\, B) \propto {\mathbb{P}(B \,|\, A)}{\mathbb{P}(A)}
  $$

\end{frame}

\begin{frame}{Take a Step Back}
  \begin{center}
    \includegraphics[width=0.95\textwidth]{./diagrams/prior.png}
  \end{center}
\end{frame}

\begin{frame}{Take a Step Back}
  \begin{center}
    \includegraphics[width=0.95\textwidth]{./diagrams/post1.png}
  \end{center}
\end{frame}

\begin{frame}{Take a Step Back}
  \begin{center}
    \includegraphics[width=0.95\textwidth]{./diagrams/postN.png}
  \end{center}
\end{frame}

\section{Introducing the Reverend Brown}

\begin{frame}{Robert Brown (1827)}

  \begin{itemize}
  \item You wish to emulate the famous botanist but with a difference.
  \item You have a camera which gives approximate co-ordinates of the
    pollen particle on the slide.
  \item You have a motor which can drive the slide in horizontal and
    vertical planes.
  \item How to track the camera to minimize the particle's distance
    from the centre of the microscope's field of vision?
  \end{itemize}

\end{frame}

\begin{frame}{Mathematical Model}
  We can model the pollen's motion as

  \begin{block}{Equations of Motion}
    $$
    \begin{aligned}
      \frac{\mathrm{d}^2 x_1}{\mathrm{d}t^2} &= \omega_1(t) \\
      \frac{\mathrm{d}^2 x_2}{\mathrm{d}t^2} &= \omega_2(t)
    \end{aligned}
    $$
  \end{block}

  Writing $x_3 = \mathrm{d}x_1 / \mathrm{d}t$ and
  $x_4 = \mathrm{d}x_2 / \mathrm{d}t$ this becomes

  \begin{block}{Matrix Form}
  $$
  \frac{\mathrm{d}}{\mathrm{d}t}\begin{bmatrix}x_1 \\ x_2 \\ x_3 \\ x_4\end{bmatrix} =
  \begin{bmatrix}
    0 & 0 & 1 & 0 \\
    0 & 0 & 0 & 1 \\
    0 & 0 & 0 & 0 \\
    0 & 0 & 0 & 0
  \end{bmatrix}
  \begin{bmatrix}x_1 \\ x_2 \\ x_3 \\ x_4\end{bmatrix} +
  \begin{bmatrix}
    0 & 0 \\
    0 & 0 \\
    1 & 0 \\
    0 & 1
  \end{bmatrix}
  \begin{bmatrix}\omega_1 \\ \omega_2\end{bmatrix}
  $$
  \end{block}

\end{frame}

\begin{frame}{}

  \begin{block}{Discretizing at $0, \Delta t, 2\Delta t, \ldots $}
    $$
    \begin{bmatrix}x^{(k)}_1 \\ x^{(k)}_2 \\ x^{(k)}_3 \\ x^{(k)}_4\end{bmatrix} =
    \begin{bmatrix}
      1 & 0 & \Delta t & 0 \\
      0 & 1 & 0        & \Delta t \\
      0 & 0 & 1        & 0 \\
      0 & 0 & 0        & 1
    \end{bmatrix}
    \begin{bmatrix}x^{(k-1)}_1 \\ x^{(k-1)}_2 \\ x^{(k-1)}_3 \\ x^{(k-1)}_4\end{bmatrix} +
    \mathbf{q}_k
    $$
  \end{block}

  \begin{block}{In vector notation}
    $$
    \mathbf{x}_k = \mathbf{A} \mathbf{x}_{k_1} + \mathbf{q}_k
    $$
  \end{block}

\end{frame}

\section{Bayes for Pollen}

\begin{frame}{}

  A similar but lengthier derivation gives us the following algorithm

\end{frame}

\end{document}

