\documentclass[presentation]{beamer}
\usepackage[utf8]{inputenc}
\usepackage[T1]{fontenc}
\usepackage{fixltx2e}
\usepackage{graphicx}
\usepackage{grffile}
\usepackage{longtable}
\usepackage{wrapfig}
\usepackage{rotating}
\usepackage[normalem]{ulem}
\usepackage{amsmath}
\usepackage{textcomp}
\usepackage{amssymb}
\usepackage{capt-of}
\usepackage{hyperref}
\usepackage{listings}
\usepackage{color}
\usepackage{verse}
\RequirePackage{fancyvrb}
\DefineVerbatimEnvironment{verbatim}{Verbatim}{fontsize=\scriptsize}
\usepackage[style=alphabetic]{biblatex}
\usetheme{Frankfurt}
\author{Dominic Steinitz}
\date{Monday 17 July 17}
\title{Bayesian Change Point Detection}
\hypersetup{
 pdfauthor={Dominic Steinitz},
 pdftitle={Bayesian Change Point Detection},
 pdfkeywords={},
 pdfsubject={Bayesian change point analysis of UK / South Korea trade statistics},
 pdflang={English}}

\lstnewenvironment{haskelL}[1][]
      {\lstset{language=haskell}\lstset{escapeinside={(*@}{@*)},
          basicstyle=\ttfamily\scriptsize,
          keywordstyle=\color{blue}\ttfamily,
          stringstyle=\color{red}\ttfamily,
          commentstyle=\color{brown}\ttfamily,
          language=haskell, label= , caption= ,
          captionpos=b, numbers=none}}
      {}

\lstnewenvironment{ASM}[1][]
{\lstset{language={[x86masm]Assembler}}\lstset{
      basicstyle=\ttfamily\small,
      keywordstyle=\color{brown}\ttfamily,
      stringstyle=\color{red}\ttfamily,
      commentstyle=\color{blue}\ttfamily,
      label= ,caption= ,
      captionpos=b,
      numbers=none}}
  {}

\begin{document}

\maketitle
\begin{frame}{Outline}
\tableofcontents
\end{frame}


\section{Introduction}

\begin{frame}{April 2013}

  \begin{itemize}
    \item
        It would be fantastic if someone could
        investigate Levent's suggestion "Of course,
        implementations can take advantage of the
        underlying CPU's native floating-point
        abs/sign functions if available as well,
        avoiding explicit tests at the Haskell code;
        based on the underlying platform"
      \item
        Otherwise we'll just end up adding an extra test
        and everyone's code will run a little bit slower.
      \item Colleague points out in December 2016
        that this really slows their code down.
    \end{itemize}

\end{frame}

\begin{frame}{A Little Learning}

  \poemtitle{An Essay on Criticism}
  \settowidth{\versewidth}{Fired at first sight with what the Muse imparts,}
  \begin{verse}[\versewidth]
    A little learning is a dangerous thing; \\
    Drink deep or taste not the Pierian spring. \\
    Fired at first sight with what the Muse imparts, \\
    In fearless youth we tempt the heights of Arts;
  \end{verse}

  \begin{itemize}
    \item
      Fearlessly assumed x87 FPU is being used for
      floating-point operations
    \item All that needs to be done is emit an
      fabs instruction.
    \end{itemize}

\end{frame}

\begin{frame}{Resources}
  \begin{itemize}
    \item
      \url{https://ghc.haskell.org/trac/ghc/wiki/Newcomers} e.g. "fast rebuilding"
    \item
      folks on \#ghc are super helpful
    \item
      Create a ticket and get helpful comments.
      \url{https://ghc.haskell.org/trac/ghc/ticket/13212}
    \end{itemize}
\end{frame}

\section{Floating Point}
\begin{frame}{Take a Step Back}
  \begin{center}
    \includegraphics[width=0.95\textwidth]{./diagrams/1280px-Float_example.svg.png}
  \end{center}
\end{frame}

\section{Code}
\label{sec:orgheadline9}
\subsection{2.1}
\label{sec:orgheadline8}

\begin{frame}[fragile,label={sec:orgheadline7}]{\texttt{libraries/base/GHC/Float.hs}}

\begin{haskelL}
  abs x | x == 0 = 0 -- handles (-0.0)
        | x > 0 = x
        | otherwise = negateFloat x
\end{haskelL}
\end{frame}

\begin{frame}[fragile]{Generated X86}
\lstset{
  basicstyle=\ttfamily\small,
  keywordstyle=\color{brown}\ttfamily,
  stringstyle=\color{red}\ttfamily,
  commentstyle=\color{blue}\ttfamily,
  language={[x86masm]Assembler},
  label= ,caption= ,
  captionpos=b,
  numbers=left}

\begin{lstlisting}
        ;; Make xmm1 contain 0
        xorps %xmm1,%xmm1
        ;; Compare
        ucomiss %xmm1,%xmm0
        ;; Jump if unordered i.e. if either operand
        ;; is a NaN
        jp _c41v
        ;; Jump if equal - the first branch
        je _c41w
\end{lstlisting}
\end{frame}

\begin{frame}[fragile]{Generated X86}
\lstset{
  basicstyle=\ttfamily\small,
  keywordstyle=\color{brown}\ttfamily,
  stringstyle=\color{red}\ttfamily,
  commentstyle=\color{blue}\ttfamily,
  language={[x86masm]Assembler},
  label= ,caption= ,
  captionpos=b,
  numbers=left,
  firstnumber=last}

\begin{lstlisting}
_c41v:
        ;; We get here if either any operand is a
        ;; NaN or if x is different from 0
        xorps %xmm1,%xmm1
        ucomiss %xmm1,%xmm0
        ;; Jump if above - the second branch
        ja _c41t
\end{lstlisting}
\end{frame}

\begin{frame}[fragile]{Generated X86}
\lstset{
  basicstyle=\ttfamily\small,
  keywordstyle=\color{brown}\ttfamily,
  stringstyle=\color{red}\ttfamily,
  commentstyle=\color{blue}\ttfamily,
  language={[x86masm]Assembler},
  label= ,caption= ,
  captionpos=b,
  numbers=left,
  firstnumber=last}

\begin{lstlisting}
_c41q:
        ;; We get here if x is less than 0 or ;;
	;; possibly if any operand is a NaN - the
	;; third branch
        leaq GHC.Types.F#_con_info(%rip),%rax
        movq %rax,-8(%r12)
        ;; This contains 0x80000000 so the top bit
	;; gets flipped
        movss _n41H(%rip),%xmm1
        xorps %xmm1,%xmm0
        movss %xmm0,(%r12)
        leaq -7(%r12),%rbx
        addq $-16,%rbp
        jmp *(%rbp)
\end{lstlisting}
\end{frame}

\begin{frame}[fragile]{Generated X86}
\lstset{
  basicstyle=\ttfamily\small,
  keywordstyle=\color{brown}\ttfamily,
  stringstyle=\color{red}\ttfamily,
  commentstyle=\color{blue}\ttfamily,
  language={[x86masm]Assembler},
  label= ,caption= ,
  captionpos=b,
  numbers=left,
  firstnumber=last}

\begin{lstlisting}
_c41w:
        addq $-16,%r12
        leaq GHC.Float.rationalToFloat4_closure(%rip),%rbx
        addq $-16,%rbp
        jmp *(%rbx)
_c41t:
\end{lstlisting}
\end{frame}

\begin{frame}[fragile]{rwbarton comments}
Yes, this would be worthwhile and not difficult. See

\begin{itemize}
\item
  \url{https://.../wiki/Commentary/PrimOps\#AddinganewPrimOp}
  for the steps involved
\item
  \url{https://phabricator.haskell.org/D1334} for a recent
  example.
\end{itemize}

\end{frame}

\begin{frame}[fragile]{Follow the Commentary: Amend \texttt{primops.txt.pp}}

  Add
  \begin{haskelL}
    primop FloatFabsOp "fabsFloat#" Monadic
           Float# -> Float#
  \end{haskelL}

  Why \texttt{Monadic}?

\end{frame}

\begin{frame}[fragile]{Follow the Example}

  If you see sucha \texttt{PrimOp} then in \texttt{compiler/codeGen/StgCmmPrim.hs}: Stg to C--

  \begin{haskelL}
    callishPrimOpSupported :: DynFlags -> PrimOp ->
                              Either CallishMachOp GenericOp
    callishPrimOpSupported dflags op =
      case op of
      ...
        FloatFabsOp
          | (ncg && x86ish)
              || llvm      -> Left MO_F32_Fabs
          | otherwise      -> Right $ genericFabsOp W32
      ...
  \end{haskelL}
\end{frame}

\begin{frame}[fragile]{Follow the Example: Amend the AST}

  \begin{itemize}
    \item \texttt{MO\_F32\_Fabs} doesn't exist!
    \item It's similar to the x86 sqrt instruction so let's place them near each other.
    \item In \texttt{compiler/cmm/CmmMachOp.hs}
      \begin{haskelL}
        data CallishMachOp
        = MO_F64_Pwr
        | MO_F64_Sin
        ...
        | MO_F64_Exp
        | (*@\textcolor{red}{MO\_F64\_Fabs}@*)
        | MO_F64_Sqrt
        ...
      \end{haskelL}
  \end{itemize}
\end{frame}

\begin{frame}[fragile]{What If We See \texttt{MO\_F32\_Fabs}?}
  In \texttt{compiler/llvmGen/LlvmCodeGen/CodeGen.hs} generate the llvm instruction
  \begin{haskelL}
    MO_F32_Fabs -> fsLit "llvm.fabs.f32"
  \end{haskelL}

\end{frame}

\begin{frame}[fragile]{What If We See \texttt{MO\_F32\_Fabs}?}
  In \texttt{compiler/nativeGen/X86/CodeGen.hs}, it's slightly more complicated
  \begin{haskelL}
  genCCall
    :: DynFlags
    -> Bool                     -- 32 bit platform?
    -> ForeignTarget            -- function to call
    -> [CmmFormal]        -- where to put the result
    -> [CmmActual]        -- arguments (of mixed type)
    -> NatM InstrBlock
    ...
    (PrimTarget op, [r])
      | sse2 -> case op of
          MO_F32_Fabs -> case args of
            [x] -> sse2FabsCode W32 x
            _ -> panic "... for fabs"
    ...
  \end{haskelL}
  otherwise generate a ``call'' (in assembly language) to
  \texttt{fabs} (not efficient).
\end{frame}


\begin{frame}[fragile]{What If We See \texttt{MO\_F32\_Fabs}?}
  \texttt{sse2NegCode :: Width -> CmmExpr -> NatM Register} does almost what we want so let's modify it
  \begin{haskelL}
    sse2FabsCode :: Width -> CmmExpr -> NatM InstrBlock
    sse2FabsCode w x = do
      let fmt = floatFormat w
    x_code <- getAnyReg x
    let
      const | FF32 <- fmt = CmmInt 0x7fffffff W32
            | otherwise   = CmmInt 0x7fffffffffffffff W64
    Amode amode amode_code <- memConstant (widthInBytes w) const
    tmp <- getNewRegNat fmt
    let
      code dst = x_code dst <> amode_code <> fromList [
          MOV fmt (OpAddr amode) (OpReg tmp),
          AND fmt (OpReg tmp) (OpReg dst)
          ]

    return $ code (getRegisterReg platform True (CmmLocal r))
  \end{haskelL}
\end{frame}

\begin{frame}[fragile]{What If We See \texttt{MO\_F32\_Fabs}?}

  \begin{ASM}
    movss _n2ca(%rip),%xmm1
    andps %xmm1,%xmm0
  \end{ASM}
\end{frame}

% sse2NegCode :: Width -> CmmExpr -> NatM Register
% sse2NegCode w x = do
%   let fmt = floatFormat w
%   x_code <- getAnyReg x
%   -- This is how gcc does it, so it can't be that bad:
%   let
%     const = case fmt of
%       FF32 -> CmmInt 0x80000000 W32
%       FF64 -> CmmInt 0x8000000000000000 W64
%   Amode amode amode_code <- memConstant (widthInBytes w) const
%   tmp <- getNewRegNat fmt
%   let
%     code dst = x_code dst `appOL` amode_code `appOL` toOL [
%         MOV fmt (OpAddr amode) (OpReg tmp),
%         XOR fmt (OpReg tmp) (OpReg dst)
%         ]
%   return (Any fmt code)
% \end{haskelL}

% -- Compute an expression into *any* register, adding the appropriate
% -- move instruction if necessary.
% getAnyReg :: CmmExpr -> NatM (Reg -> InstrBlock)
% getAnyReg expr = do
%   r <- getRegister expr
%   anyReg r

% What does C do?
% Amend the AST
% Generate the correct ASM
% Generate the correct llvm (easy)
% Generate C--
% Someone else has fixed Solaris(?)

% https://stackoverflow.com/questions/44630015/how-would-fabsdouble-be-implemented-on-x86-is-it-an-expensive-operation

\end{document}
